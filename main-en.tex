\documentclass[12pt, twoside]{book}
%\documentclass[12pt, oneside]{book}  % jednostranna tlac

%spravne nastavenie okrajov
\usepackage[a4paper,top=2.5cm,bottom=2.5cm,left=3.5cm,right=2cm]{geometry}
%zapnutie fontov pre UTF8 kodovanie
\usepackage[utf8]{inputenc}
\usepackage[OT1]{fontenc}

%zapnutie slovenskeho delenia slov
%a automatickych nadpisov ako Obsah, Obrázok a pod. v slovencine
\usepackage[main=english,slovak]{babel} % vypnite pre prace v anglictine!

%nastavenie riadkovania podla smernice
\linespread{1.25} % hodnota 1.25 by mala zodpovedat 1.5 riadkovaniu

% balicek na vkladanie zdrojoveho kodu
\usepackage{listings}
% ukazky kodu su cislovane ako Listing 1,2,...
% tu je Listing zmenene na Algoritmus 1,2,...
\renewcommand{\lstlistingname}{Algorithm}
% nastavenia balicka listings
% mozete pridat aj language=...
% na nastavenie najcastejsie pouzivaneho prog. jazyka
% takisto sa da zapnut cislovanie riadkov
\lstset{frame=lines}

% balicek na vkladanie obrazkov
\usepackage{graphicx}
% balicek na vkladanie celych pdf dokumentov, tu zadanie
\usepackage{pdfpages}
% balicek na spravne formatovanie URL
\usepackage{url}
% balicek na hyperlinky v ramci dokumentu
% zrusime farebne ramiky okolo liniek aby pdf
% vyzeralo rovnako ako tlacena verzia
\usepackage[hidelinks,breaklinks]{hyperref}
\usepackage{enumitem}
\usepackage{amsmath, amssymb, amsfonts, amsthm}
\usepackage{biblatex}

\addbibresource{bibliography.bib}

\newtheorem{theorem}{Theorem}[chapter]
\newtheorem{lemma}[theorem]{Lemma}
\newtheorem{conjecture}[theorem]{Conjecture}
\newtheorem{proposition}[theorem]{Proposition}
\newtheorem{problem}[theorem]{Problem}

\theoremstyle{definition}
\newtheorem*{definition}{Definition}

\theoremstyle{definition}
\newtheorem{corollary}{Corollary}[theorem]

\theoremstyle{remark}
\newtheorem*{remark}{Remark}
\newtheorem*{claim}{Claim}

\usepackage{tikz}
\usetikzlibrary{shapes.geometric}
\usepackage{esvect}
% \usetikzlibrary{external}
% \tikzexternalize[prefix=images/]

\input constants.tex

\begin{document}
\frontmatter
\pagestyle{empty}

% -------------------
% --- Obalka ------
% -------------------

\begin{center}
  \sc\large
  Comenius University in Bratislava\\
  Faculty of Mathematics, Physics and Informatics

\vfill

{\LARGE\mfnazov}\\
\mftyp
\end{center}

\vfill

{\sc\large
\noindent \mfrok\\
\mfautor
}

\cleardoublepage
% --- koniec obalky ----

% -------------------
% --- Titulný list
% -------------------

\noindent

\begin{center}
\sc
\large
  Comenius University in Bratislava\\
  Faculty of Mathematics, Physics and Informatics

\vfill

{\LARGE\mfnazov}\\
\mftyp
\end{center}

\vfill

\noindent
\begin{tabular}{ll}
Study Programme: & \program \\
Field of Study: & \mfodbor \\
Department: & \mfpracovisko \\
Supervisor: & \mfskolitel \\
% Consultant: & \mfkonzultant \\
\end{tabular}

\vfill


\noindent \mfmiesto\\
\mfautor

\cleardoublepage
% --- Koniec titulnej strany


% -------------------
% --- Zadanie z AIS
% -------------------
% v tlačenej verzii s podpismi zainteresovaných osôb.
% v elektronickej verzii sa zverejňuje zadanie bez podpisov
% v pracach v anglictine anglicke aj slovenske zadanie

\newpage
\setcounter{page}{3}
\includepdf[trim=-10mm 0mm 5mm 0mm]{images/zadanie.pdf}
\includepdf[trim=5mm 0mm -10mm 0mm]{images/zadanie-en.pdf}

% --- Koniec zadania


% -------------------
%   Poďakovanie - nepovinné
% -------------------
\newpage
\pagestyle{plain}
~

\vfill
{\bf Acknowledgments:} 
% Tu môžete poďakovať školiteľovi, prípadne
%ďalším osobám, ktoré vám s prácou nejako pomohli, poradili,
%poskytli dáta a podobne.

% --- Koniec poďakovania

% -------------------
%   Abstrakt - Slovensky
% -------------------
\newpage
\section*{Abstrakt}

% Slovenský abstrakt v rozsahu 100-500 slov, jeden odstavec. Abstrakt
% stručne sumarizuje výsledky práce. Mal by byť pochopiteľný pre bežného
% informatika. Nemal by teda využívať skratky, termíny alebo označenie
% zavedené v práci, okrem tých, ktoré sú všeobecne známe.

\paragraph*{Kľúčové slová:} 
% --- Koniec Abstrakt - Slovensky


% -------------------
% --- Abstrakt - Anglicky
% -------------------
\newpage
\section*{Abstract}

% Abstract in the English language (translation of the abstract in the
% Slovak language).
\paragraph*{Keywords:} 

% --- Koniec Abstrakt - Anglicky

% -------------------
% --- Predhovor - v informatike sa zvacsa nepouziva
% -------------------
%\newpage
%
%
%\chapter*{Preface} %
%
%Predhovor je všeobecná informácia o práci, obsahuje hlavnú charakteristiku práce
%a okolnosti jej vzniku. Autor zdôvodní výber témy, stručne informuje o cieľoch
%a význame práce, spomenie domáci a zahraničný kontext, komu je práca určená,
%použité metódy, stav poznania; autor stručne charakterizuje svoj prístup a svoje
%hľadisko.
%
% --- Koniec Predhovor


% -------------------
% --- Obsah
% -------------------

\cleardoublepage

\tableofcontents

% ---  Koniec Obsahu

% -------------------
% --- Zoznamy tabuliek, obrázkov - nepovinne
% -------------------

% \newpage

% \listoffigures
% \listoftables

% ---  Koniec Zoznamov

\mainmatter
\pagestyle{headings}

\input intro.tex

\input flows.tex
\input generation.tex
\input bounds.tex

\input conclusion.tex



%\input zaver.tex

% -------------------
% --- Bibliografia
% -------------------


\newpage

\backmatter

\thispagestyle{empty}
\clearpage

% \bibliographystyle{plain}
% \bibliography{literatura}
\printbibliography

%Prípadne môžete napísať literatúru priamo tu
%\begin{thebibliography}{5}

%\bibitem{br1} MOLINA H. G. - ULLMAN J. D. - WIDOM J., 2002, Database Systems, Upper Saddle River : Prentice-Hall, 2002, 1119 s., Pearson International edition, 0-13-098043-9

%\bibitem{br2} MOLINA H. G. - ULLMAN J. D. - WIDOM J., 2000 , Databasse System implementation, New Jersey : Prentice-Hall, 2000, 653s., ???

%\bibitem{br3} ULLMAN J. D. - WIDOM J., 1997, A First Course in Database Systems, New Jersey : Prentice-Hall, 1997, 470s.,

%\bibitem{br4} PREFUSE, 2007, The Prefuse visualization toolkit,  [online] Dostupné na internete: <http://prefuse.org/>

%\bibitem{br5} PREFUSE Forum, Sourceforge - Prefuse Forum,  [online] Dostupné na internete: <http://sourceforge.net/projects/prefuse/>

%\end{thebibliography}

%---koniec Referencii

% -------------------
%--- Prilohy---
% -------------------

%Nepovinná časť prílohy obsahuje materiály, ktoré neboli zaradené priamo  do textu. Každá príloha sa začína na novej strane.
%Zoznam príloh je súčasťou obsahu.
%

\end{document}
