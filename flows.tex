\chapter{Flows} % chapter* je necislovana kapitola

In this chapter, we aim to introduce all concepts and current knowledge about flows and related graph theory terms. Mainly, we explain Chebyshev nowhere-zero flows, which
have arisen in the Student Science Conference paper of the author \cite{svk}. Moreover, we present a concept of two smaller flows as a sufficient condition for a graph to have
a larger nowhere-zero flow and some Chebyshev flow, also stated in that paper. Then we connect this idea to the theory of group connectivity of a graph established by Jaeger
et al. \cite{group_connectivity} and $k$-closures and $k$-bases introduced by Seymour \cite{seymour} in the paper, where he proved the $6$-flow theorem.

% TODO definition of generalised Blanuša snarks

\section{Integral nowhere-zero flows}

In this thesis, we consider only unoriented unweighted simple graphs $\Gamma=(V,E)$. By a \emph{dart} we represent an edge with an additional orientation. This means, for any edge $e=uv\in E$ we have two darts $\vv{uv}$, $\vv{vu}$ belonging to that edge. By a \emph{dart-set} of $S\subseteq V$ we mean the set of darts with initial vertex in the set $S$.

\begin{definition}
    Let $\Gamma=(V, E)$ be a graph and $k\geq 2$ be an integer. A \emph{$k$-flow} on $\Gamma$ is an assignment $\varphi\colon D(V)\to \{-(k-1), -(k-2),\dots, k-2, k-1\}$ of flow values to the darts satisfying
    $\varphi(\vv{uv})=-\varphi(\vv{vu})$ for any dart $\vv{uv}\in D(V)$ and
	\begin{equation}
		\sum_{\vv d\in D(v)} \mkern-10mu\varphi\left(\vv d\right) = 0 \label{eq:conservation}
	\end{equation}
    for each vertex $v$ of $G$. The equality \eqref{eq:conservation} is often referred to as the \textit{flow conservation constraint}. Moreover, a \emph{nowhere-zero $k$-flow} (or NZ $k$-flow) is a $k$-flow such that no flow value is zero, i.e. $\varphi\left(\vv d\right)\neq 0$ for any $\vv d\in D(V)$.
\end{definition}

We also establish a notation $\varphi(e)$ for $e=uv\in E$, where $\varphi(e)=\max\{\varphi(\vv{uv}), \varphi(\vv{vu})\}$.

The flow conservation constraint \eqref{eq:conservation} is generalisable to any subset of vertices. Hence, having a graph $\Gamma$ with a bridge $e$, the interpretation of the flow conservation constraint for one of the components of $\Gamma - e$ results in the following observation.

\begin{claim}
    If $e\in E$ is a bridge in $\Gamma$, then $\varphi(e)=0$. Moreover, $\Gamma$ allows a NZ $k$-flow for no natural $k$.
\end{claim}

Therefore, NZ flows are examined only on bridgeless graphs.

We can clearly see that graph with an NZ $k$-flow also admits an NZ $(k+1)$-flow. That means, graphs with an NZ $2$-flow are somehow the ``simplest'' ones and the existence of NZ flows relates with some complexity of a graph. So we define a flow number, which is a rate of this complexity. 

\begin{definition}
	Let $\Gamma$ be a bridgeless graph. A \emph{flow number of $\Gamma$} is
	\begin{equation*}
		\Phi(\Gamma) := \min\{k\mid\exists\text{NZ } k\text{-flow on }\Gamma\}.\label{eq:flow_number}
	\end{equation*}
\end{definition}

However, it turns out that this rate of complexity achieves only few values.

\begin{theorem}[$6$-flow theorem] \emph{\cite[p. 133]{seymour}}
    There exists an NZ $6$-flow on any bridgeless graph.\label{th:6_flow}
\end{theorem}

Seymour constructed this NZ $6$-flow from two simpler flows from the fact, that if $a\in\{0,\pm 1\}$, $b\in\{0,\pm 1,\pm 2\}$ and not both $a, b$ are zero, then $3a+b\in\{\pm 1, \pm 2, \pm 3, \pm 4, \pm 5\}$.

\begin{lemma} \emph{\cite[p. 132]{seymour}}
    For any bridgeless graph, there exist a $2$-flow $\varphi_2$ and a $3$-flow $\varphi_3$ such that for each edge $e$, at least one value from $\varphi_2(e), \varphi_3(e)$ is non-zero.\label{lem:2_flow_3_flow_seymour}
\end{lemma}

On the other hand, there is no known graph with $\Phi(\Gamma)=6$, all known graphs have $\Phi(\Gamma)\leq 5$. Therefore, the conjecture about the NZ $5$-flow by Tutte still holds unresolved.

\begin{conjecture}[$5$-flow conjecture] \emph{\cite[p. 83]{tutte}}
    There exists an NZ $5$-flow on any bridgeless graph.\label{conj:5_flow}
\end{conjecture}

As for many problems in the graph theory, it is sufficient to explore the problem only on the smaller subset of graphs. There exists a folklore reduction of nowhere-zero flows to $3$-regular (or cubic) graphs.
\begin{proposition}
    There exists an NZ $k$-flow on any bridgeless graph if and only if there exists an NZ $k$-flow on any cubic bridgeless graph.\label{prop:reduction}
\end{proposition}

Hence, by considering only cubic graphs, we still solve the universal problem. Moreover, in terms of finding a counterexample to Tutte's $5$-flow conjecture, we can also avoid $3$-edge-colourable ones and limit research to the cubic graphs, that are not $3$-edge-colourable, as a corollary of the next proposition.

\begin{proposition} \emph{\cite[pp. 160, 161]{diestel}}
    A cubic graph has an NZ $4$-flow if and only if it is $3$-edge-colourable.
\end{proposition}

The class of type II cubic graphs is widely used, so its members have special name.
\begin{definition}
    A \emph{snark} is a bridgeless cubic graph, which has no cycles of lengths $3$ and $4$, is cyclically $4$-edge connected (at least $4$ edges must be removed from the graph to get two components containing a cycle) and is not $3$-edge-colourable.
\end{definition}

\section{Chebyshev flows and sufficient flow-pairs}

There were some attempts to generalise nowhere-zero flows to real numbers and more dimensions. By generalising integer nowhere-zero flows to real numbers, researchers found out, that restricting only zero flow value is not sufficient, since any flow consisting of non-zero flow values can be made ``smaller'' and then, the notion of flow number disappears. Therefore, we need to determine a larger interval of real values forbidden as flow values, and hence, one-dimensional real nowhere-zero flows are allowed to use flow values from the interval $[1,r-1]$. Then the first attempts of generalisation real flows to more dimensions \cite{TODO} tried to pick only vectors with the Euclidean norm from the mentioned interval $[1,r-1]$. It worked, but it did not have such nice properties, as we would expect. However, changing a norm turned out to be a good idea \cite{svk}, so we will consider only Chebyshev multidimensional nowhere-zero flows. 

\begin{definition}
    Let $\Gamma=(V, E)$ be a bridgeless graph, $d$ be a positive integer and $r\geq 2$ be a real constant. A \emph{$d$-dimensional Chebyshev nowhere-zero $r$-flow} (or $d$D ChNZ $r$-flow) satisfies the conservation constraint \eqref{eq:conservation} and the condition $1\leq\|\varphi(e)\|_\infty\leq r-1$ for each edge $e\in E$.
\end{definition}

The flow number can be defined in a similar way to integral nowhere-zero flows. The only problem is, that the set of real numbers does not need to contain a minimal element, so formally correct definition would use infimum instead of minimum. However, the infimum of the flow number candidates is always feasible \cite{TODO}, so we can define the flow number straightforward with minimum.

\begin{definition}
	Let $\Gamma$ be a bridgeless graph and $d$ be a positive integer. A \emph{$d$-dimensional Chebyshev flow number} of $\Gamma$ is
	\begin{equation*}
		\Phi_d^\infty(\Gamma) := \min\{r\mid\exists d\text{D ChNZ }\text{-flow on }\Gamma\}.
	\end{equation*}
\end{definition}

Now we recall the Seymour's flows from Lemma \ref{lem:seymour_2_flow_3_flow} -- a $2$-flow and a $3$-flow with a property, that no edge has both flow values equal to zero. By simply using the value of the $2$-flow as the first coordinate and the value of the $3$-flow as the second coordinate, we get a Chebyshev nowhere-zero flow. Moreover, this flow gives us a nice bound on the Chebyshev flow number.

\begin{proposition} \emph{\cite[p. 343]{svk}}
    Each bridgeless graph has a 2D ChNZ $3$-flow and hence, $\Phi_2^\infty(\Gamma)\leq 3$.\label{prop:chebyshev_upper_seymour}
\end{proposition}

Again, no graph with $\Phi_2^\infty(\Gamma)=3$ is known. Futhermore, there is no known graph with $\Phi_2^\infty(\Gamma)>5/2$. Therefore, the next conjecture is interesting to examine.

\begin{conjecture} \emph{\cite[p. 344]{svk}}
    Each bridgeless graph has a 2D ChNZ $5/2$-flow and hence, $\Phi_2^\infty(\Gamma)\leq 5/2$.\label{conj:chebyshev_upper_seymour}
\end{conjecture}

And similarly as for integral flows, the problem can be reduced to cubic graphs and then to snarks, since $3$-edge-colourable cubic graphs have small flow number.

\begin{theorem} \emph{\cite[p. 343]{svk}}
    A cubic graph has a 2D ChNZ $2$-flow if and only if it is $3$-edge-colourable.\label{th:2_chnzf_iff_3_col}
\end{theorem}

As an NZ $6$-flow and a 2D ChNZ $3$-flow can be constructed from two simpler flows, something similar can be made for smaller flow numbers. However, the existence of two simpler flows is only a sufficient condition to the existence of the original flow, the condition is not necessary. Hence we will use a term \emph{$p/q$-sufficient flow-pair} instead of a former \emph{$(p,q)$-circulation decomposition}.

\begin{definition}
        Let $p\leq q$ be the positive integers. A \emph{$p/q$-sufficient flow-pair} of a graph $\Gamma$ is a $2$-flow $\varphi_2$ and a $(p+q+1)$-flow $\varphi_{p+q+1}$ such that whenever $\varphi_2(e)$ is zero, the value $\varphi_{p+q+1}(e)$ is at least $q$.
\end{definition}

One possible ambiguity of this definition is, that since $p/q$ is written as a fraction, the $2/6$-sufficient flow-pair can be mixed up with the $1/3$-sufficient flow-pair. However, this is not a bug, this is a feature, since we can WLOG assume $p,q$ relatively prime.

\begin{lemma}
A bridgeless graph $\Gamma$ with a $dp/dq$-sufficient flow-pair also has a $p/q$-sufficient flow-pair and vice versa.
\end{lemma}

We mentioned, that the $p/q$-sufficient flow-pair is a sufficient condition to the existence of some nowhere-zero flows. Those flows are somehow related with the value $p/q$ and that is the reason to write it as a fraction in the definition of the sufficient flow-pair,

\begin{proposition} \emph{\cite[p. 344]{svk}}
    Consider a bridgeless graph $\Gamma$ with a $p/q$-sufficient flow-pair. Then, there are a 2D ChNZ $(2+p/q)$-flow and an NZ $(4+2p/q)$-flow on $\Gamma$.\label{prop:nzf_from_sufficient_flow_pair}
\end{proposition}

Note that if there is a graph, which is a counterexample to Tutte's 5-flow conjecture, it should not have the $1/2$-sufficient flow-pair. Hence it would be interesting to classify graphs with $1/2$-sufficient flow-pair. By a computer, it was checked that this flow-pair exists for snarks up to roughly $30$ vertices, so it looks like the $1/2$-sufficient flow-pair exists always.

\begin{conjecture} \emph{\cite[p. 346]{svk}}
    Each bridgeless graph has a $1/2$-sufficient flow-pair.
\end{conjecture}

% TODO is it necessary?
% \begin{lemma}
%	Let $p<q$ be positive integers. Consider a snark $G$ with a $(p,q)$-circulation decomposition. Then the set of edges $e$ with a property $\varphi_2(e)\neq 0$ is a $2$-factor of $G$.\label{lem:2_circ_snark_2_factor}
%\end{lemma}

\section{Closures and bases}

\begin{definition} \cite[p. 132]{seymour}
	Let $\Gamma=(V,E)$ be a graph and $S\subseteq E$ its edge subset. A \emph{$k$-closure} $\langle S\rangle_{k}$ of $S$ is a mimimal $T$ such that $S\subseteq T\subseteq E$ and for any circuit $\mathcal C\not\subseteq T$ of $G$, the size of $\mathcal C\cap T^c$ is strictly greater than $k$.
\end{definition}

\begin{definition} \cite[p. 7]{group_connectivity_enumeration}
	Let $\Gamma=(V,E)$ be a graph. Then $S\subseteq E$ is a \emph{$k$-base} of $\Gamma$ if $\langle S\rangle_k = E$.
\end{definition}

\begin{claim}
	Any spanning tree of a graph is its inclusion-minimal $1$-base.
\end{claim}

\begin{lemma} \cite[p. 133]{seymour}
	For any $3$-connected graph $\Gamma$ there exists its even-factor which is also its $2$-base.
\end{lemma}
% TODO add connection to Seymour's proof

\begin{lemma} \cite[p. 134]{seymour}
	For any $3$-connected cubic graph $\Gamma=(V,E)$ there exists a vertex partition $E=E_1\cup E_2$ such that $E_1,E_2$ are its $1$- and $2$-bases, respectively.
\end{lemma}
% TODO add connection to Seymour's proof

\section{Group connectivity}

% TODO introduction, connection with choosability and list properties

% TODO definition

% TODO generalised first Seymour's lemma

% TODO discuss generalisation of the second Seymour's lemma
