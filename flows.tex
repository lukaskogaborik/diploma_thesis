\chapter{Flows} % chapter* je necislovana kapitola

In this chapter, we aim to introduce all concepts and current knowledge about flows and related graph theory terms. Mainly, we explain Chebyshev nowhere-zero flows, which
have arisen in the Student Science Conference paper of the author \cite{svk}. Moreover, we present a concept of two smaller flows as a sufficient condition for a graph to have
a larger nowhere-zero flow and some Chebyshev flow, also stated in that paper. Then we connect this idea to the theory of group connectivity of a graph established by Jaeger
et al. \cite{group_connectivity} and $k$-closures and $k$-bases introduced by Seymour \cite{seymour} in the paper, where he proved the $6$-flow theorem.

% TODO definition of generalised Blanuša snarks

% TODO rewrite G -> \Gamma
\section{Integral nowhere-zero flows}

In this thesis, we consider only unoriented unweighted simple graphs $\Gamma=(V,E)$. By a \emph{dart} we represent an edge with an additional orientation. This means, for any edge $e=uv\in E$ we have two darts $\vv{uv}$, $\vv{vu}$ belonging to that edge. By a \emph{dart-set} of $S\subseteq V$ we mean the set of darts with initial vertex in the set $S$.

\begin{definition}
    Let $\Gamma=(V, E)$ be a graph and $k\geq 2$ be an integer. A \emph{$k$-flow} on $\Gamma$ is an assignment $\varphi\colon D(V)\to \{-(k-1), -(k-2),\dots, k-2, k-1\}$ of flow values to the darts satisfying
    $\varphi(\vv{uv})=-\varphi(\vv{vu})$ for any dart $\vv{uv}\in D(V)$ and
	\begin{equation}
		\sum_{\vv d\in D(v)} \mkern-10mu\varphi\left(\vv d\right) = 0 \label{eq:conservation}
	\end{equation}
    for each vertex $v$ of $G$. The equality \eqref{eq:conservation} is often referred to as the \textit{flow conservation constraint}. Moreover, a \emph{nowhere-zero $k$-flow} (or NZ $k$-flow) is a $k$-flow such that no flow value is zero, i.e. $\varphi\left(\vv d\right)\neq 0$ for any $\vv d\in D(V)$.
\end{definition}

We also establish a notation $\varphi(e)$ for $e=uv\in E$, where $\varphi(e)=\max\{\varphi(\vv uv), \varphi(\vv vu)\}$.

The flow conservation constraint \eqref{eq:conservation} is generalisable to any subset of vertices. Hence, having a graph $\Gamma$ with a bridge $e$, the interpretation of the flow conservation constraint for one of the components of $\Gamma - e$ results in the following observation.

\begin{claim}
    If $e\in E$ is a bridge in $\Gamma$, then $\varphi(e)=0$. Moreover, $\Gamma$ allows a NZ $k$-flow for no natural $k$.
\end{claim}

Therefore, NZ flows are examined only on bridgeless graphs.

We can clearly see that graph with an NZ $k$-flow also admits an NZ $(k+1)$-flow. That means, graphs with an NZ $2$-flow are somehow the ``simplest'' ones and the existence of NZ flows relates with some complexity of a graph. So we define a flow number, which is a rate of this complexity. 

\begin{definition}
	Let $\Gamma$ be a bridgeless graph. A \emph{flow number of $\Gamma$} is
	\begin{equation*}
		\Phi(\Gamma) := \min\{k\mid\exists\text{NZ } k\text{-flow on }\Gamma\}.\label{eq:flow_number}
	\end{equation*}
\end{definition}

However, it turns out that this rate of complexity achieves only few values.

\begin{theorem}[$6$-flow theorem] \emph{\cite[p. 133]{seymour}}
    There exists an NZ $6$-flow on any bridgeless graph.\label{th:6_flow}
\end{theorem}

Seymour constructed this NZ $6$-flow from two simpler flows.

\begin{lemma} \emph{\cite[p. 132]{seymour}}
    For any bridgeless graph, there exist a $2$-flow $\varphi_2$ and a $3$-flow $\varphi_3$ such that for each edge $e$, at least one value from $\varphi_2(e), \varphi_3(e)$ is non-zero.\label{lem:2_flow_3_flow_seymour}
\end{lemma}

% TODO check
\begin{conjecture} \emph{\cite[p. 83]{tutte}}
    There exists a $5$-NZF on any bridgeless graph.\label{conj:5_flow}
\end{conjecture}

% TODO check, remove bipartite
\begin{proposition} \emph{\cite[pp. 160, 161]{diestel}}
    A cubic graph has a $4$-NZF if and only if it is $3$-edge-colourable. Moreover, it has a $3$-NZF if and only if it is bipartite.
\end{proposition}

% TODO check
\begin{definition}
    A \emph{snark} is a bridgeless cubic graph, which has no cycles of lengths $3$ and $4$, is cyclically $4$-edge connected (at least $4$ edges must be removed from the graph to get two components containing a cycle) and is not $3$-edge-colourable.
\end{definition}

\section{Chebyshev nowhere-zero flows and sufficient flow-pairs}

% TODO rewrite with darts, minimum, erase Manhattan norm
\begin{definition}
	Let $G$ be a bridgeless graph and $d$ be a positive integer. A \emph{$d$-dimensional Manhattan flow number} of $G$ and a \emph{$d$-dimensional Chebyshev flow number} of $G$ are
	\begin{equation*}
		\Phi_d^\text{M}(G) := \inf\{r\mid\exists(r, d)\text{-MNZF on }G\}\text{ and } \Phi_d^\text{Ch}(G) := \inf\{r\mid\exists(r, d)\text{-ChNZF on }G\},
	\end{equation*}
 respectively.
\end{definition}

% TODO check
\begin{proposition}
    Each bridgeless graph has a $(3, 2)$-ChNZF.\label{prop:manhattan_upper_seymour}
\end{proposition}

% TODO rewrite to Chebyshev
\begin{theorem}
    A cubic graph has a $(2, 2)$-MNZF if and only if it is $3$-edge-colourable.\label{th:2_mnzf_iff_3_col}
\end{theorem}

% TODO rewrite with darts, sufficient flow-pair, value between, add reasoning
\begin{definition}
        Let $p\leq q$ be the positive integers. A \emph{$(p,q)$-circulation decomposition} of a graph $G$ is a $2$-circulation $(o_2,\varphi_2)$ and a $(p+q+1)$-circulation $(o_{p+q+1}, \varphi_{p+q+1})$ such that whenever $\varphi_2(e)$ is zero, the value $|\varphi_{p+q+1}(e)|$ is either $q$ or $p+q$.
\end{definition}

% TODO check
\begin{lemma}
A bridgeless graph $G$ with a $(dp,dq)$-circulation decomposition also has a $(p,q)$-circulation decomposition.\label{lem:circ_rel_prime}
\end{lemma}

% TODO check
\begin{proposition}
    Consider a bridgeless graph $G$ with a $(p,q)$-circulation decomposition. Then, there are $\left(2+\frac pq,2\right)$-ChNZF and $\left(4+\frac {2p}q, 1\right)$-NZF on $G$.\label{prop:nzf_from_two_circulations_generalised}
\end{proposition}

% TODO check
\begin{conjecture}
    For each bridgeless graph there exists a $(1, 2)$-circulation decomposition.
\end{conjecture}

% TODO is it necessary?
\begin{lemma}
	Let $p<q$ be positive integers. Consider a snark $G$ with a $(p,q)$-circulation decomposition. Then the set of edges $e$ with a property $\varphi_2(e)\neq 0$ is a $2$-factor of $G$.\label{lem:2_circ_snark_2_factor}
\end{lemma}

\section{Closures and bases}

\begin{definition} \cite[p. 132]{seymour}
	Let $\Gamma=(V,E)$ be a graph and $S\subseteq E$ its edge subset. A \emph{$k$-closure} $\langle S\rangle_{k}$ of $S$ is a mimimal $T$ such that $S\subseteq T\subseteq E$ and for any circuit $\mathcal C\not\subseteq T$ of $G$, the size of $\mathcal C\cap T^c$ is strictly greater than $k$.
\end{definition}

\begin{definition} \cite[p. 7]{group_connectivity_enumeration}
	Let $\Gamma=(V,E)$ be a graph. Then $S\subseteq E$ is a \emph{$k$-base} of $\Gamma$ if $\langle S\rangle_k = E$.
\end{definition}

\begin{claim}
	Any spanning tree of a graph is its inclusion-minimal $1$-base.
\end{claim}

\begin{lemma} \cite[p. 133]{seymour}
	For any $3$-connected graph $\Gamma$ there exists its even-factor which is also its $2$-base.
\end{lemma}
% TODO add connection to Seymour's proof

\begin{lemma} \cite[p. 134]{seymour}
	For any $3$-connected cubic graph $\Gamma=(V,E)$ there exists a vertex partition $E=E_1\cup E_2$ such that $E_1,E_2$ are its $1$- and $2$-bases, respectively.
\end{lemma}
% TODO add connection to Seymour's proof

\section{Group connectivity}

% TODO introduction, connection with choosability and list properties

% TODO definition

% TODO generalised first Seymour's lemma

% TODO discuss generalisation of the second Seymour's lemma
